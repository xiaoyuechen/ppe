\documentclass{article}
\usepackage[utf8]{inputenc}
\usepackage{graphicx}
\usepackage{listings}
\usepackage{booktabs}

\title{PPE Final Project}
\author{Hannah Atmer and Xiaoyue Chen \\ Team 6}
\date{October 2021}

\begin{document}

\maketitle

% Your grade on your project will be a combination of how challenging optimizations you choose, how well you implement them, and how well you analyze the results.

\section{Optimized motionVectorSearch}

Originally takes 95.22\% of runtime using 4K image sizes. This equals around 100 seconds per frame.
We determined that the bottleneck is the code that compares the block to all other possible blocks in the window in order to find the best possible match for where that block moved to.
We decided to use the GPU to speed up this comparison, and we used the map-reduce algorithm to divide the work amongst GPU cores, collect the results, and reduce the results to find the best match.

3 different types of index mapping

% TODO detailed description of algorithm

MotionVectorSearch is only applied to frames that are market (P) in the execution statistics.

After GPU parallelization, motionVectorSearch takes 0.0\% of computation time. The motionVectorSearch function still takes 600ms which is the data transfer time.

% TODO picture of kernel

\section{Optimized lowPass}

We parallelized convertRGBtoYCbCr and lowPass with openMP.

% TODO picture of parallel lowpass

\section{Optimized loadImage}

In the original program, the loadImage function takes 21.12\% of computation time, which is 860 milliseconds in absolute time.


The loadImage function converts the frame data from Tiff format to RGB by inverting the rows.

After optimization the loadImage function takes 780 milliseconds in absolute time.

\section{Compiler Flags \& Cache Optimization}
For the optimized version, we used the -O3 compiler optimization flag. This flag automatically uses SIMD instructions where possible, among other automatic assembly optimizations.

We reorderd for loops to optimize cache access. % TODO improve

\section{Analysis}

absolute time of function

\begin{table}[h!]
  \centering
  \begin{tabular}{lll}
    \toprule
    Function & Original (ms) & Optimized (ms) \\
    \midrule
    loadImage & 860 & 780 \\

    convertRGBtoYCbCr & 1000 & 30 \\

    lowPass & 2000 & 100 \\

    motionVectorSearch & 400000 & 500 \\

    \bottomrule
  \end{tabular}
  \caption{Optimization Results}
  \label{tab:results}
\end{table}

data on runtime computation percentage
-> but doesn't account for memory access/transfer time

call graphs before and after




\end{document}

%%% Local Variables:
%%% mode: latex
%%% TeX-master: t
%%% End:
