\documentclass{article} \usepackage[utf8]{inputenc}
% \usepackage{tabularx}

\title{Lab 1} \author{Hannah Atmer and Xiaoyue Chen \\ Team 6}
\date{September 2021}

\begin{document}

\maketitle
\section{Performance Estimates}
\subsection{Our Hardware}

\begin{tabular}{l | l}
Architecture:                    & x86\_64 \\
CPU op-mode(s):                  & 32-bit, 64-bit \\
CPU(s):                          & 12 \\
Model name:                      & Intel(R) Core(TM) i7-8750H CPU \@
                                   2.20GHz \\
CPU family:                      & 6 \\
Model:                           & 158 \\
Thread(s) per core:              & 2 \\
Core(s) per socket:              & 6 \\
Socket(s):                       & 1 \\
CPU max MHz:                     & 4100.0000 \\
CPU min MHz:                     & 800.0000 \\
Cache line:                      & 64 B \\
L1d cache:                       & 192 KiB (6 instances) \\
L1i cache:                       & 192 KiB (6 instances) \\
L2 cache:                        & 1.5 MiB (6 instances) \\
L3 cache:                        & 9 MiB (1 instance) \\
\end{tabular}





Your estimates of the performance benefits of each of the
optimizations and how you came up with them.



\begin{tabularx}{0.8\textwidth} {
  | >{\centering\arraybackslash}X
  | >{\centering\arraybackslash}X
  | >{\centering\arraybackslash}X | }
\hline
Optimization & Predicted Impact & Why \\
\hline

B  & 1.0  & Baseline  \\
\hline
C  & x3 while memory bounded & Raise the sloping part of roof and move ridge to the left, i.e. improve memory-bounded performance  \\
\hline
S  & less than x4 since lack of cache affinity will make the program memory bounded & SIMD allows for x4 FLOPS/sec. This raises the flat part of roof, i.e. increase maximum possible CPU-bounded performance x4, and moves the ridge to the right, meaning higher operational intensity is required to achieve this level of performance \\
\hline
P  &  at most xN, where N is number of cores  & Parallelizing loops will improve performance depending on how many cores used and on Amdahl's law \\
\hline
C + S  & x4  & SIMD makes the program much more operationally intensive. Performance will be CPU bounded since the cache is not a bottleneck.   \\
\hline
C + P  & at most xN  & parallelization increases operational intensity \\
\hline
S + P  & x4 - x4N  & parallel SIMD will be more operationally intensive than just SIMD  \\
\hline
C + S + P  & item  & item  \\
\hline


\end{tabularx}


\section{Performance Achieved}

\subsection{Measured performance results from implementing the
  optimizations}

\begin{tabularx}{1.3\textwidth} {
  | >{\centering\arraybackslash}X
  | >{\centering\arraybackslash}X
  | >{\centering\arraybackslash}X
  | >{\centering\arraybackslash}X
  | >{\centering\arraybackslash}X
  | >{\centering\arraybackslash}X
  | >{\centering\arraybackslash}X
  | >{\centering\arraybackslash}X
  | >{\centering\arraybackslash}X
  | >{\centering\arraybackslash}X  | }
\hline
& B & C & S & P & C + S & C + P & S + P & C + S + P\\
\hline

Measured Time  & item  & item & item & item & item & item & item & item \\
\hline
Predicted Speedup  & item  & item & item & item & item & item & item & item  \\
\hline
Measured Speedup  & item  & item & item & item & item & item & item & item \\
\hline



\end{tabularx}

\subsection{Parallel scaling of your implementation}

\begin{tabularx}{0.8\textwidth} {
  | >{\centering\arraybackslash}X
  | >{\centering\arraybackslash}X
  | >{\centering\arraybackslash}X
  | >{\centering\arraybackslash}X | }
\hline
& 1 Core & 2 Cores & 4 Cores \\
\hline

P  & item  & item & item \\
\hline
C + P  & item  & item & item \\
\hline
S + P & item  & item  & item\\
\hline
C + S + P  & item  & item  & item\\
\hline


\end{tabularx}



\section{Discussion}

A discussion of the measured results and how they differ from your
predictions.

A discussion of what you learned about these optimizations from
implementing them and measuring the results.

Comment on any unexpected or odd results.

Comments on the difficulty of each optimization.

Comment on what the compiler did for you.

\section{Lab comments}
Any feedback on the lab itself.

\end{document}

%%% Local Variables:
%%% mode: latex
%%% TeX-master: t
%%% End:
